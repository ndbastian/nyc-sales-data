\documentclass[]{article}
\usepackage{lmodern}
\usepackage{amssymb,amsmath}
\usepackage{ifxetex,ifluatex}
\usepackage{fixltx2e} % provides \textsubscript
\ifnum 0\ifxetex 1\fi\ifluatex 1\fi=0 % if pdftex
  \usepackage[T1]{fontenc}
  \usepackage[utf8]{inputenc}
\else % if luatex or xelatex
  \ifxetex
    \usepackage{mathspec}
  \else
    \usepackage{fontspec}
  \fi
  \defaultfontfeatures{Ligatures=TeX,Scale=MatchLowercase}
\fi
% use upquote if available, for straight quotes in verbatim environments
\IfFileExists{upquote.sty}{\usepackage{upquote}}{}
% use microtype if available
\IfFileExists{microtype.sty}{%
\usepackage{microtype}
\UseMicrotypeSet[protrusion]{basicmath} % disable protrusion for tt fonts
}{}
\usepackage[margin=1in]{geometry}
\usepackage{hyperref}
\hypersetup{unicode=true,
            pdftitle={Predicting Real Estate Sales Using Machine Learning and Spatial Dependence},
            pdfborder={0 0 0},
            breaklinks=true}
\urlstyle{same}  % don't use monospace font for urls
\usepackage{graphicx,grffile}
\makeatletter
\def\maxwidth{\ifdim\Gin@nat@width>\linewidth\linewidth\else\Gin@nat@width\fi}
\def\maxheight{\ifdim\Gin@nat@height>\textheight\textheight\else\Gin@nat@height\fi}
\makeatother
% Scale images if necessary, so that they will not overflow the page
% margins by default, and it is still possible to overwrite the defaults
% using explicit options in \includegraphics[width, height, ...]{}
\setkeys{Gin}{width=\maxwidth,height=\maxheight,keepaspectratio}
\IfFileExists{parskip.sty}{%
\usepackage{parskip}
}{% else
\setlength{\parindent}{0pt}
\setlength{\parskip}{6pt plus 2pt minus 1pt}
}
\setlength{\emergencystretch}{3em}  % prevent overfull lines
\providecommand{\tightlist}{%
  \setlength{\itemsep}{0pt}\setlength{\parskip}{0pt}}
\setcounter{secnumdepth}{0}
% Redefines (sub)paragraphs to behave more like sections
\ifx\paragraph\undefined\else
\let\oldparagraph\paragraph
\renewcommand{\paragraph}[1]{\oldparagraph{#1}\mbox{}}
\fi
\ifx\subparagraph\undefined\else
\let\oldsubparagraph\subparagraph
\renewcommand{\subparagraph}[1]{\oldsubparagraph{#1}\mbox{}}
\fi

%%% Use protect on footnotes to avoid problems with footnotes in titles
\let\rmarkdownfootnote\footnote%
\def\footnote{\protect\rmarkdownfootnote}

%%% Change title format to be more compact
\usepackage{titling}

% Create subtitle command for use in maketitle
\newcommand{\subtitle}[1]{
  \posttitle{
    \begin{center}\large#1\end{center}
    }
}

\setlength{\droptitle}{-2em}
  \title{Predicting Real Estate Sales Using Machine Learning and Spatial
Dependence}
  \pretitle{\vspace{\droptitle}\centering\huge}
  \posttitle{\par}
\subtitle{Boosting Random Forest Predictive Accuracy Using Spatial Lags}
  \author{}
  \preauthor{}\postauthor{}
  \date{}
  \predate{}\postdate{}

\setlength{\parindent}{4em}
\setlength{\parskip}{0em}

\begin{document}
\maketitle

{
\setcounter{tocdepth}{2}
\tableofcontents
}
\section{Introduction}\label{introduction}

\subsection{What is Economic
Exclusion?}\label{what-is-economic-exclusion}

Income inequality may be a defining challenge of our time. Researchers
at the Urban Institute (Solomon Greene and Lei 2016) recently identified
the socio-economic phenomenon of ``Economic Exclusion'' as one
compelling explanation for the recent rise in inequality in the US. As
discussed by Zuk (2015), ``Neighborhoods change slowly, but over time
are becoming more segregated by income, due in part to macro-level
increases in income inequality''. Vulnerable
populations--disproportionately communities of color, immigrants,
refugees, and women--who are displaced by localized economic prosperity
enter into a gradual cycle of diminished access to good jobs, good
schools, health care facilities, public spaces, etc. Such systematic
denial causes enduring and self-reinforcing poverty over the course
years and even generations, gradually entrenching income inequality and
general unrest.

One way to practically combat economic exclusion is to focus on
preventing displacement, however, detecting gentrification at an early
enough stage can be a daunting task. When an area experiences economic
growth, increased housing demands and subsequent affordability pressures
can lead to evictions of low-income families and small businesses.
Government agencies and nonprofits tend to intervene once displacement
is already underway, and after-the-fact interventions can be costly and
ineffective. There are a host of preemptive actions that can be deployed
to stem divestment and ensure that existing residents benefit from new
investments. Not unlike medical treatment, early detection is key to
success. Consequently, in 2016, the Urban Institute put forth a call for
research into the creation of ``neighborhood-level early warning and
response systems that can help city leaders and community advocates get
ahead of neighborhood changes'' (2016).

(M. Chapple Karen; Zuk 2016) To be included in the ``motivation''
section of my thesis. Not about predictive modeling, but is a very
recent overview of the application of predictive gentrification models

\subsection{How Can Machine Learning
Help?}\label{how-can-machine-learning-help}

Predictive modeling using spatial dependence has been employed
extensively in recent years, notably in Crime Prediction (Almanie 2015).
However, a key deficiency of many spatial models are their use of
arbitrarily defined geographic regions, such as zip codes, political
districts, police precincts, state lines, neighborhoods, etc. which
diminish and obscure potentially valuable insights. Worse yet, many
predictive models ignore spatial dependence, violating one of the basic
tenets of geography: the direct relationship between distance and
likeness (Miller 2007).

\subsection{Our Contribution}\label{our-contribution}

This paper explores novel techniques to predict gentrification in the
pursuit of combating displacement and economic exclusion. Modern
techniques of data mining, machine learning and predictive modeling are
applied to data sets describing property values and sale prices in New
York City. We demonstrate that the incorporation of spatial lags, i.e.,
variables created from physically proximate observations, can improve
the predictive accuracy of machine learning models above and beyond both
non-spatial models as well as models which incorporate data aggregated
at arbitrary geographic regions such as zip codes.

\section{Literature Review}\label{literature-review}

The literature review for this paper reviews the concept of Economic
Displacement as it has been addressed in academia, primarily in relation
to the study of gentrification. We also examine ``mass appraisal
techniques'', which are automated analytical techniques used for valuing
large numbers of real estate properties. Finally, we will briefly
examine machine learning as it relates to the problem of predicting
gentrification and/or Economic Displacement.

\subsection{How Has Economic Displacement Been Addressed in the
Past?}\label{how-has-economic-displacement-been-addressed-in-the-past}

Economic Displacement has been intertwined with the study of
gentrification since shortly after the latter became academically
relevant in the 1960's. The term ``gentrification'' was first used by
Ruth Glass in 1964 to described the ``gentry'' in low income
neighborhoods in London. Gentrification was originally understood as a
``tool of revitalization for declining neighborhoods'' (Zuk 2015),
however, in 1979 Phillip Clay made the distinction between two types of
revitalization: ``incumbent upgrading'' and ``gentrification'', noting
that Economic Displacement was the negative consequence of the latter
(Clay 1979). Today, the term has evolved to describe ``a spatial
organization and re-organization of human dwelling and activity'' (Zuk
2015). Specific to cities, gentrification is thought of as ``the
transformation of a working-class or vacant area of the central city
into middle-class residential or commercial use'' (Lees 2008).

Studies of gentrification and displacement generally take two approaches
in the literature: supply-side and demand-side, or ``the flows of
capital versus flows of people to neighborhoods'', respectively (Zuk
2015). Supply side arguments for gentrification tend to focus on
``private capital investment, public policies, and public investments''
(Zuk 2015). (Smith 1979) argued that the return of capital from the
suburbs to the city drives gentrification. He describes a ``political
economy of capital flows into urban areas'' (Zuk 2015) as largely
responsible for both the positive and negative consequences of
gentrification. According to (Dreier 2004), public policies that have
been linked to increased Economic Displacement have been, among others,
automobile-oriented transportation infrastructure spending and mortgage
interest tax deductions for home owners.

More recently, income inequality has been explored as a major
contributor to Economic Displacement, specifically, ``higher
compensation in the top quintile and the lack of jobs for the bottom
quintile'' (Reardon 2011); (Watson 2009). The concentration of wealth
allows ``certain households to sort themselves according to their
preferences -- and control local political processes that continue
exclusion'' (Reardon 2011). This results in a self-reinforcing feedback
loop where wealthier households influence public policy toward their
self interest. Gentrification prediction tools could be used to help
break such feedback loops through early identification and intervention.
Reardon (2011) also argues that ``were income inequality to stop rising,
the number of segregated neighborhoods would decline.''

Many studies conclude that gentrification in most forms leads to
exclusionary economic displacement, however, Zuk (2015) characterizes
the results of many recent studies as ``mixed, due in part to
methodological shortcomings''. In this paper, we attempt to further the
understanding of gentrification prediction by demonstrating a technique
to better predict real estate sales in New York City.

\subsection{A Review of Mass Appraisal
Techniques}\label{a-review-of-mass-appraisal-techniques}

Much of the research on predicting real estate prices has been in
service of creating mass appraisal models. Mass appraisal models are
most commonly used by local governments for the purpose of collecting
taxes from property owners. Mass appraisal models share many
characteristics with predictive machine learning models, in that they
are data-driven, standardized methods that employ statistical testing
(Eckert 1990). A variation on mass appraisal models are the ``automated
valuation models'' (AVM), which use ``often the same methodological
framework of mass appraisal\ldots{} a statistical model and a large
amount of property data to estimate the market value of an individual
property or portfolio of properties'' (d'Amato 2017).

Scientific mass appraisal models date back to 1936 with the reappraisal
of St.~Paul, Minnesota (Silverherz 1936). Since that time, and
accelerated with the advent of computers, much statistical research has
been done relating property values and rent prices to various
characteristics of those properties, including characteristics of their
surrounding area. Multiple regression analysis (MRA) has been the most
common set of statistical tools used in mass appraisal, including
Maximum Likelihood, Weighted Least Squares, and the most popular,
Ordinary Least Squares (OLS) (d'Amato 2017). The primary drawbacks of
MRA techniques are ``excessive multicollinearity among attributes'' and
``spatial autocorrelation among residuals'' (d'Amato 2017). Another
group of models that seek to correct for spatial dependence are known as
Spatial Auto Regressive (SAR) models, chief among them the Spatial Lag
Model, which aggregates weighted summaries of nearby properties in order
to create independent regression variables (d'Amato 2017).

Hedonic regression models generally seek to break down the price of a
good based on the intrinsic and extrinsic components. Koschinsky (2012)
is a recent and thorough discussion of parametric hedonic regression
techniques. Some of the variables included in Koschinsky's models are
derived from nearby properties, similar to the technique used in this
paper, and these variables were found to be predictive. The real estate
hedonic model as defined by Koschinsky describes the price of a property
as:

\[
\begin{aligned}
 P_i = P(S_i, N_i, L_i)
\end{aligned}
\]

Where \(P_i\) represents the price of house \(i\), which is a composite
good comprised of a vector of structural characteristics \(S\), a vector
of social and neighborhood characteristics \(N\), and a vector of
locational characteristics \(L\). Specifically, the model calculates
spatial lags on properties of interest using neighboring properties
within 1,000 feet of a sale. The derived variables include
characteristics like average age, quantity of poor condition homes,
percent of homes with electric heating, construction grade, etc. within
1,000 feet of the property in question. Koschinsky found that in all
cases, ``the relation between a home's price and the average price of
its neighboring homes is characterized by positive spatial
autocorrelation'' meaning that homes near each other were typically
similar to each other and priced accordingly. Koschinsky concluded that
locational characteristics should be valued at least as much ``if not
more'' than important structural characteristics.

As recently as 2015, much research has dealt with mitigating the
drawbacks of MRA, including the use of multi-level hierarchical models.
Fotheringham (2015) explored the combination of Geographically Weighted
Regression (GWR) with time-series forecasting to predict home prices
over time. GWR is a variation on OLS that allows for ``adaptive
bandwidths'' of local data to be included, i.e., for each estimate, the
number of data points included varies and can be optimized using
cross-validation.

Automated valuation modeling got a legal update in the aftermath of the
2008 financial crisis by way of the The Dodd Frank Act. In particular,
the Title XIV, subtitle F distinguishes the ``appraisal'' process from
automated valuation modelling, and reorganized both (d'Amato 2017). The
Act asserts that appraisal, or valuation conducted by a human being,
cannot be replaced by AVM. At current, AVM is ``increasingly adaptable
in describing real estate market behavior'' but has yet to supersede the
importance and necessity of local information and human evaluation.

\subsection{Has Machine Learning Been Applied to this Problem
Before?}\label{has-machine-learning-been-applied-to-this-problem-before}

Both Mass Appraisal techniques and Automated Valuation Modeling seek to
predict real estate prices using data and statistical methods, however,
traditional techniques typically fall short of reality. This is because
property valuation is inherently a ``chaotic'' process that does not
lend itself to binary or linear analysis (Zuk 2015). The value of any
given property is a complex combination of perceived value and
speculation. The value of any building or plot of land belongs to a rich
network where decisions about and perceptions of neighboring properties
influence the final market value. Guan et al. (2014) compared
traditional MRA techniques to alternative ``data mining techniques''
resulting in ``mixed results''. However, as Helbich (2013) states,
hedonic pricing models ``can be improved in two ways: (a) Through novel
estimation techniques, and (b) by ancillary structural, locational, and
neighborhood variables on the basis of Geographic Information System
(GIS)''. Recent research generally falls into these two buckets: better
analysis algorithms and/or better data.

In the ``better data'' category, researchers have been striving to
introduce new independent variables to increase the accuracy of
predictive models. Dietzell (2014) successfully used internet search
query data provided by Google Trends to serve as a sentiment indicator
and improve commercial real estate forecasting models. Pivo and Fisher
(2011) examined the effects of walkability on property values and
investment returns. Pivo found that on a 100-point scale, a 10-point
increase in walkability increased property investment values by up to
9\%.

Research into better prediction algorithms do not necessarily happen at
the exclusion of ``better data''. For example, Fu (2014) created a
prediction algorithm, called ``ClusRanking'', for real estate in
Beijing, China. ClusRanking first estimates neighborhood characteristics
using taxi cab traffic vector data, specifically as they relate to
accessibility to ``business areas''. Then, the algorithm performs a
rank-ordered prediction of investment returns segmented into five
categories. Similar to Koschinsky (2012), though less formally stated,
Fu (2014) thought of a property's value as a composite of individual,
peer and zone characteristics. In the predictive model, Fu includes
characteristics of the neighborhood (individual), the values of its
nearby properties (peer), and the prosperity of the affiliated latent
business area (zone) based on taxi cab data (Fu 2014).

Several other recent studies compare various ``advanced'' statistical
techniques either to other advanced techniques or to traditional ones.
Most studies conclude that the advanced, non-parametric techniques
outperform traditional parametric techniques. Kontrimasa (2011) compares
the accuracy of linear regression against the SVM technique and found
the latter to outperform. Schernthanner H. (2016) compared traditional
linear regression techniques to several techniques such as krigging
(stochastic interpolation) and random forest. They concluded that the
more advanced techniques, particularly random forest, are sound and more
accurate when compared to traditional statistical methods. Guan et al.
(2014) compared three different approaches to defining spatial
neighbors: a simple radius technique, a k-nearest neighbors technique
using only distance and a k-nearest neighbors technique using all
attributes. Interestingly, the location-only KNN models performed best,
although by a slight margin. Park (2015) developed several housing price
prediction models based on machine learning algorithms including C4.5,
RIPPER, Naive Bayesian, and AdaBoost. By comparing the models'
classification accuracy performance, the experiments demonstrate that
the RIPPER algorithm, based on accuracy, consistently outperformed the
other models in the performance of housing price prediction. Rafiei
(2016) employed a restricted boltzmann machine (neural network with back
propagation) to predict the sale price of residential condos in Tehran,
Iran. Rather than focusing on predictive performance, their paper
focuses on computational efficiency. A non-mating genetic algorithm is
used for dimensionality reduction. The paper concludes that two primary
strategies help in this regard: weighting property sales by temporal
proximity (sales which happened closer in time are more important), and
also using a learner to accelerate the recognition of important
features. The paper compares this technique to several other common
neural network approaches and finds that while not necessarily the only
way to get the best answer, it is the fastest way to get to the best
answer.

Finally, it should be noted that many studies, whether exploring
advanced techniques, new data, or both, rely on aggregation of data by
some arbitrary boundary. For example, Turner and Snow (2001) predicted
gentrification in the Washington, D.C. metro area by ranking census
tracts in terms of development. K. Chapple (2009) created a
gentrification ``early warning system'' by identifying low income census
tracts in central city locations. Barry Bluestone \& Chase Billingham
(2010) analyzed 42 census block groups near rail stations in 12 metro
areas across the United States, studying changes between 1990 and 2000
for neighborhood socioeconomic and housing characteristics. All of these
studies, and many more, relied on aggregation of data at the
census-tract or census-block level. In contrast, this paper compares
boundary-aggregation techniques (specifically, aggregating by zip codes)
to spatial-lag techniques and finds the spatial lag techniques to
generally outperform.

\subsection{sample citations}\label{sample-citations}

Sample Citation: (Antipov and Pokryshevskaya 2012) (see: Antipov and
Pokryshevskaya 2012, 33--35; also Antipov and Pokryshevskaya 2012, ch.~1
and \emph{passim})

A minus sign (-) before the @ will suppress mention of the author in the
citation. This can be useful when the author is already mentioned in the
text:

Antipov says blah (2012).

You can also write an in-text citation, as follows:

Antipov and Pokryshevskaya (2012) says blah.

\section{Methodology}\label{methodology}

\subsection{Data}\label{data}

\subsection{Algorithm}\label{algorithm}

Random Forrest has several advantages over traditional geographic
weighted regression, amoung them:

\begin{enumerate}
\def\labelenumi{\arabic{enumi}.}
\tightlist
\item
  Ability to handle large amounts of categorical data without much
  pre-processing
\item
  Ability to model in spite of missing values in data
\item
  Eliminated colinearity as a concern
\item
  Allows for the introduction of many more variables without requiring
  penalty for additional predictors
\item
  Works relatively fast and can be parallelized
\end{enumerate}

\subsection{Model Diagnostics}\label{model-diagnostics}

\section{Results}\label{results}

\subsection{Probability of Sale Model}\label{probability-of-sale-model}

\subsection{Sale Price Model}\label{sale-price-model}

\subsection{Using the Models in
Practice}\label{using-the-models-in-practice}

\section{Conclusions and Future
Research}\label{conclusions-and-future-research}

\subsection{Future Research}\label{future-research}

\subsection{Conclusion}\label{conclusion}

\section*{References}\label{references}
\addcontentsline{toc}{section}{References}

\hypertarget{refs}{}
\hypertarget{ref-Almanie2015}{}
Almanie, R.; Lor, T.; Mirza. 2015. ``Crime Prediction Based on Crime
Types and Using Spatial and Temporal Criminal Hotspots.''
\emph{International Journal of Data Mining \& Knowledge Management
Process (IJDKP)} 5 (4).

\hypertarget{ref-antipov12}{}
Antipov, Evgeny A., and Elena B. Pokryshevskaya. 2012. ``Mass Appraisal
of Residential Apartments: An Application of Random Forest for Valuation
and a Cart-Based Approach for Model Diagnostics.'' \emph{Expert Systems
with Applications}.

\hypertarget{ref-Pollack2010}{}
Barry Bluestone \& Chase Billingham, Stephanie Pollack \&. 2010.
``Maintaining Diversity in America's Transit-Rich Neighborhoods: Tools
for Equitable Neighborhood Change.'' \emph{New England Community
Developments, Federal Reserve Bank of Boston}, 1--6.

\hypertarget{ref-Chapple2009}{}
Chapple, Karen. 2009. ``Mapping Susceptibility to Gentrification: The
Early Warning Toolkit.'' \emph{Berkeley, CA: Center for Community
Innovation.}

\hypertarget{ref-Chapple2016}{}
Chapple, Miriam, Karen; Zuk. 2016. ``Forewarned: The Use of Neighborhood
Early Warning Systems for Gentrification and Displacement.''
\emph{Cityscape: A Journal of Policy Development and Research} 18 (3).

\hypertarget{ref-Clay1979}{}
Clay, Phillip L. 1979. \emph{Neighborhood Renewal: Middle-Class
Resettlement and Incumbent Upgrading in American Neighborhoods}.
Lexington Books.

\hypertarget{ref-Dietzell2014}{}
Dietzell, Nicole; Schäfers, Marian Alexander; Braun. 2014.
``Sentiment-Based Commercial Real Estate Forecasting with Google Search
Volume Data.'' \emph{Journal of Property Investment \& Finance,} 32 (6):
540--69.

\hypertarget{ref-Dreier2004}{}
Dreier, John; Swanstrom, Peter; Mollenkopf. 2004. \emph{Place Matters:
Metropolitics for the Twenty-First Century.} University Press of Kansas.

\hypertarget{ref-Springer2017}{}
d'Amato, Tom, Maurizio; Kauko, ed. 2017. \emph{Advances in Automated
Valuation Modeling}. Springer International Publishing.

\hypertarget{ref-Eckert1990}{}
Eckert, J. K. 1990. \emph{Property Appraisal and Assessment
Administration}. Chicago, IL.: International Association of Assessing
Officers.

\hypertarget{ref-Fotheringham2015}{}
Fotheringham, R; Yao, A.S.; Crespo. 2015. ``Exploring, Modelling and
Predicting Spatiotemporal Variations in House Prices.'' \emph{The Annals
of Regional Science} 54.

\hypertarget{ref-Fu2014}{}
Fu, Yanjie; et al. 2014. \emph{Exploiting Geographic Dependencies for
Real Estate Appraisal: A Mutual Perspective of Ranking and Clustering}.
Proceedings of the 20th ACM SIGKDD international conference on Knowledge
discovery; data mining.

\hypertarget{ref-Geltner2017}{}
Geltner, David, and Alex Van de Minne. 2017. ``Do Different Price Points
Exhibit Different Investment Risk and Return Commercial Real Estate.''
Real Estate Research Institute.

\hypertarget{ref-Guan2014}{}
Guan, Jian, Donghui Shi, Jozef M. Zurada, and Alan S. Levitan. 2014.
``Analyzing Massive Data Sets: An Adaptive Fuzzy Neural Approach for
Prediction, with a Real Estate Illustration.'' \emph{Journal of
Organizational Computing and Electronic Commerce} 24 (1). Taylor \&
Francis: 94--112.
doi:\href{https://doi.org/10.1080/10919392.2014.866505}{10.1080/10919392.2014.866505}.

\hypertarget{ref-Helbich2013}{}
Helbich, et al., Marco. 2013. ``Boosting the Predictive Accuracy of
Urban Hedonic House Price Models Through Airborne Laser Scanning.''
\emph{Computers, Environment and Urban Systems} 39: 81--92.

\hypertarget{ref-Johnson2007}{}
Johnson, Ken, Justin Benefield, and Jonathan Wiley. 2007. ``The
Probability of Sale for Residential Real Estate.'' \emph{Journal of
Housing Research} 16 (2): 131--42.
doi:\href{https://doi.org/10.5555/jhor.16.2.0234g75800h5k8x6}{10.5555/jhor.16.2.0234g75800h5k8x6}.

\hypertarget{ref-Kontrimasa2011}{}
Kontrimasa, Antanas, Vilius; Verikasb. 2011. ``The Mass Appraisal of the
Real Estate by Computational Intelligence.'' \emph{Applied Soft
Computing}.

\hypertarget{ref-Koschinsky2012}{}
Koschinsky, J. et al. 2012. ``The Welfare Benefit of a Home's Location:
An Empirical Comparison of Spatial and Non-Spatial Model Estimates.''
\emph{Journal of Geographical Systems} 10109.

\hypertarget{ref-Lees2008}{}
Lees, Tom; Wyly, Loretta; Slater. 2008. ``Gentrification.'' \emph{Growth
and Change} 39 (3): 536--39.
doi:\href{https://doi.org/10.1111/j.1468-2257.2008.00443.x}{10.1111/j.1468-2257.2008.00443.x}.

\hypertarget{ref-Miller2015}{}
Miller, J.; Aspinall, J.; Franklin. 2007. ``Incorporating Spatial
Dependence in Predictive Vegetation Models.'' \emph{Ecological
Modelling} 202 (3): 225--42.

\hypertarget{ref-Park2015}{}
Park, Jae Kwon, Byeonghwa; Bae. 2015. ``Using Machine Learning
Algorithms for Housing Price Prediction: The Case of Fairfax County,
Virginia Housing Data.'' \emph{Expert Systems with Applications} 42 (6):
2928--34.

\hypertarget{ref-Pivo2011}{}
Pivo, Gary, and Jeffrey D. Fisher. 2011. ``The Walkability Premium in
Commercial Real Estate Investments.'' \emph{Real Estate Economics} 39
(2): 185--219.
doi:\href{https://doi.org/10.1111/j.1540-6229.2010.00296.x}{10.1111/j.1540-6229.2010.00296.x}.

\hypertarget{ref-Rafiei2016}{}
Rafiei, Hojjat, Mohammad Hossein; Adeli. 2016. ``A Novel Machine
Learning Model for Estimation of Sale Prices of Real Estate Units.''
\emph{Journal of Construction Engineering and Management} 142 (2).

\hypertarget{ref-Reardon2011}{}
Reardon, Kendra, Sean F.; Bischoff. 2011. ``Income Inequality and Income
Segregation.'' \emph{American Journal of Sociology}.

\hypertarget{ref-Schernthanner2016}{}
Schernthanner H., Gonschorek J., Asche H. 2016. ``Spatial Modeling and
Geovisualization of Rental Prices for Real Estate Portals.''
\emph{Computational Science and Its Applications} 9788.

\hypertarget{ref-Silverherz1936}{}
Silverherz, J. D. 1936. ``The Assessment of Real Property in the United
States.'' \emph{Albany: J.B. Lyon Co. Printers}.

\hypertarget{ref-Smith1979}{}
Smith, Neil. 1979. ``Toward a Theory of Gentrification a Back to the
City Movement by Capital, Not People.'' \emph{Journal of the American
Planning Association} 45 (4). Routledge: 538--48.
doi:\href{https://doi.org/10.1080/01944367908977002}{10.1080/01944367908977002}.

\hypertarget{ref-urban2016}{}
Solomon Greene, Molly Scott, Rolf Pendall, and Serena Lei. 2016. ``Open
Cities: From Economic Exclusion to Urban Inclusion.'' \emph{Urban
Institue Brief}, June. Urban Institue Brief.

\hypertarget{ref-Turner2001}{}
Turner, Margery Austin, and Christopher Snow. 2001. \emph{Leading
Indicators of Gentrification in d.C. Neighborhoods}.

\hypertarget{ref-Watson2009}{}
Watson, Tara. 2009. ``Inequality and the Measurement of Residential
Segregation by Income in American Neighborhoods.'' \emph{Review of
Income and Wealth}.

\hypertarget{ref-Zuk2015}{}
Zuk, Miriam; et al. 2015. ``Gentrification, Displacement and the Role of
Public Investment: A Literature Review.''


\end{document}
