\documentclass[]{article}
\usepackage{lmodern}
\usepackage{amssymb,amsmath}
\usepackage{ifxetex,ifluatex}
\usepackage{fixltx2e} % provides \textsubscript
\ifnum 0\ifxetex 1\fi\ifluatex 1\fi=0 % if pdftex
  \usepackage[T1]{fontenc}
  \usepackage[utf8]{inputenc}
\else % if luatex or xelatex
  \ifxetex
    \usepackage{mathspec}
  \else
    \usepackage{fontspec}
  \fi
  \defaultfontfeatures{Ligatures=TeX,Scale=MatchLowercase}
\fi
% use upquote if available, for straight quotes in verbatim environments
\IfFileExists{upquote.sty}{\usepackage{upquote}}{}
% use microtype if available
\IfFileExists{microtype.sty}{%
\usepackage{microtype}
\UseMicrotypeSet[protrusion]{basicmath} % disable protrusion for tt fonts
}{}
\usepackage[margin=1in]{geometry}
\usepackage{hyperref}
\hypersetup{unicode=true,
            pdftitle={A Predictive Model for Real Estate Sales Using Machine Learning and Spatial Dependence},
            pdfborder={0 0 0},
            breaklinks=true}
\urlstyle{same}  % don't use monospace font for urls
\usepackage{graphicx,grffile}
\makeatletter
\def\maxwidth{\ifdim\Gin@nat@width>\linewidth\linewidth\else\Gin@nat@width\fi}
\def\maxheight{\ifdim\Gin@nat@height>\textheight\textheight\else\Gin@nat@height\fi}
\makeatother
% Scale images if necessary, so that they will not overflow the page
% margins by default, and it is still possible to overwrite the defaults
% using explicit options in \includegraphics[width, height, ...]{}
\setkeys{Gin}{width=\maxwidth,height=\maxheight,keepaspectratio}
\IfFileExists{parskip.sty}{%
\usepackage{parskip}
}{% else
\setlength{\parindent}{0pt}
\setlength{\parskip}{6pt plus 2pt minus 1pt}
}
\setlength{\emergencystretch}{3em}  % prevent overfull lines
\providecommand{\tightlist}{%
  \setlength{\itemsep}{0pt}\setlength{\parskip}{0pt}}
\setcounter{secnumdepth}{0}
% Redefines (sub)paragraphs to behave more like sections
\ifx\paragraph\undefined\else
\let\oldparagraph\paragraph
\renewcommand{\paragraph}[1]{\oldparagraph{#1}\mbox{}}
\fi
\ifx\subparagraph\undefined\else
\let\oldsubparagraph\subparagraph
\renewcommand{\subparagraph}[1]{\oldsubparagraph{#1}\mbox{}}
\fi

%%% Use protect on footnotes to avoid problems with footnotes in titles
\let\rmarkdownfootnote\footnote%
\def\footnote{\protect\rmarkdownfootnote}

%%% Change title format to be more compact
\usepackage{titling}

% Create subtitle command for use in maketitle
\newcommand{\subtitle}[1]{
  \posttitle{
    \begin{center}\large#1\end{center}
    }
}

\setlength{\droptitle}{-2em}
  \title{A Predictive Model for Real Estate Sales Using Machine Learning and
Spatial Dependence}
  \pretitle{\vspace{\droptitle}\centering\huge}
  \posttitle{\par}
\subtitle{Using Spatial Lags to Create Geospatial Machine Learning Predictive
Models}
  \author{}
  \preauthor{}\postauthor{}
  \date{}
  \predate{}\postdate{}

\setlength{\parindent}{4em}
\setlength{\parskip}{0em}

\begin{document}
\maketitle

{
\setcounter{tocdepth}{2}
\tableofcontents
}
\section{Introduction}\label{introduction}

\subsection{What is Economic
Exclusion?}\label{what-is-economic-exclusion}

Income inequality may be a defining challenge of our time. Researchers
at the Urban Institute (Solomon Greene and Lei 2016) recently identified
the socio-economic phenomenon of ``Economic Exclusion'' as one
compelling explanation for the recent rise in inequality in the US. As
discussed by Zuk (2015), ``Neighborhoods change slowly, but over time
are becoming more segregated by income, due in part to macro-level
increases in income inequality''. Vulnerable
populations--disproportionately communities of color, immigrants,
refugees, and women--who are displaced by localized economic prosperity
enter into a gradual cycle of diminished access to good jobs, good
schools, health care facilities, public spaces, etc. Such systematic
denial causes enduring and self-reinforcing poverty over the course
years and even generations, gradually entrenching income inequality and
general unrest.

One way to practically combat economic exclusion is to focus on
preventing displacement, however, detecting gentrification at an early
enough stage can be a daunting task. When an area experiences economic
growth, increased housing demands and subsequent affordability pressures
can lead to evictions of low-income families and small businesses.
Government agencies and nonprofits tend to intervene once displacement
is already underway, and after-the-fact interventions can be costly and
ineffective. There are a host of preemptive actions that can be deployed
to stem divestment and ensure that existing residents benefit from new
investments. Not unlike medical treatment, early detection is key to
success. Consequently, in 2016, the Urban Institute put forth a call for
research into the creation of ``neighborhood-level early warning and
response systems that can help city leaders and community advocates get
ahead of neighborhood changes'' (2016).

(Chapple 2016) To be included in the ``motivation'' section of my
thesis. Not about predictive modeling, but is a very recent overview of
the application of predictive gentrification models

\subsection{How Can Machine Learning
Help?}\label{how-can-machine-learning-help}

Predictive modeling using spatial dependence has been employed
extensively in recent years, notably in Crime Prediction (Almanie 2015).
However, a key deficiency of many spatial models are their use of
arbitrarily defined geographic regions, such as zip codes, political
districts, police precincts, state lines, neighborhoods, etc. which
diminish and obscure potentially valuable insights. Worse yet, many
predictive models ignore spatial dependence, violating one of the basic
tenets of geography: the direct relationship between distance and
likeness (Miller 2007).

\subsection{Our Contribution}\label{our-contribution}

This paper explores novel techniques to predict gentrification in the
pursuit of combating displacement and economic exclusion. Modern
techniques of data mining, machine learning and predictive modeling are
applied to data sets describing property values and sale prices in New
York City. We demonstrate that the incorporation of spatial lags, i.e.,
variables created from physically proximate observations, can improve
the predictive accuracy of machine learning models above and beyond both
non-spatial models as well as models which incorporate data aggregated
at arbitrary geographic regions such as zip codes.

\section{Literature Review}\label{literature-review}

\subsection{How Has Economic Displacement Been Addressed in the
Past?}\label{how-has-economic-displacement-been-addressed-in-the-past}

Research on Economic Displacement dates back to the 1970s, occuring in
direct reaction to the urban renewal period in US cities (Zuk 2015).

Gentrification as a concept came into being during the 1950's and
1960's, and was first used by Glass in 1964 to described the ``gentry''
in low income neighborhoods in London. The modern conception of
gentrification is a spatial organization and re-organiztion of human
dwelling and activity. Specific to cities, gentrification is thought of
as ``the transformation of a working-class or vacant area of the central
city into middle-class residential or commercial use'' (Loretta Lees,
Slater, and Wyly 2008).

Smith (1979) argues that the return of capital from the suburbs to the
city drives gentrification; the change in neighborhoods is the spatial
manifestation of the restructuring of capital through shifting land
values and housing development. Smith (1979) sees individual gentrifiers
as important, but places a greater emphasis on a broader nexus of actors
-- developers, builders, mortgage lenders, government agencies, real
estate agents -- that make up the full political economy of capital
flows into urban areas.

Economic segregation has increased steadily since the 1970s, with a
brief respite in the 1990s, and is related closely to racial segregation
(i.e., income segregation is growing more rapidly among black families
than white) (Fischer et al. 2004; Fry and Taylor 2015; P. Jargowsky
2001; Lichter, Parisi, and Taquino 2012; Reardon and Bischoff 2011;
Watson 2009; Yang and Jargowsky 2006)

A range of studies have found that living in poor neighborhoods
negatively impacts residents, particularly young people, who are more
likely than their counterparts in wealthier neighborhoods to participate
in and be victims of criminal activity, experience teen pregnancy, drop
out of high school, and perform poorly in school among a multitude of
other negative outcomes (Crane 1991; Ellen and Turner 1997; Galster
2010; P. A. Jargowsky 1997; Jencks et al. 1990; Ludwig et al. 2001;
Sampson, Morenoff, and GannonRowley 2002; Sharkey 2013)

More recent topics covering Displacement include the relationship
between gentrification and to public investment such as transit
infrastructure.

Today, the overarching debate has generally drawn a line between the
flows of capital versus flows of people to neighborhoods. This
dichotomous narrative has spawned many analyses focused on either
production and supply-side or consumption, demand-side catalysts. Flows
of capital focus on profit-seeking and the work of broader economic
forces to make inner city areas profitable for in-movers.

But we also now understand that neighborhood income segregation within
metropolitan areas is influenced mostly by income inequality, in
particular, higher compensation in the top quintile and the lack of jobs
for the bottom quintile (Reardon and Bischoff 2011; Watson 2009). Income
inequality leads to income segregation because higher incomes, supported
by housing policy, allow certain households to sort themselves according
to their preferences -- and control local political processes that
continue exclusion (Reardon and Bischoff 2011). Other explanatory
factors include disinvestment in urban areas, suburban investment and
land use patterns, and the practices generally of government and the
underwriting industry (Hirsch 1983; Levy, McDade, and Dumlao Bertumen
2011). But were income inequality to stop rising, the number of
segregated neighborhoods would decline (Reardon and Bischoff 2011,
Watson 2009).

Zuk characterizes the results of these studies as ``mixed, due in part
to methodological shortcomings''. Many studies conclude that
gentrification in most forms leads to exclusionary economic
displacement.

Government policies shape free markets and preferences, as well as
respond to them. Thus, transportation policies favoring the automobile,
discrimination and redlining in early federal home ownership policies,
mortgage interest tax deductions for home owners, and other urban
policies have actively shaped or reinforced patterns of racial and
economic segregation, while severely constraining choices for
disadvantaged groups (Dreier, Mollenkopf, and Swanstrom 2004).

African American - White segregation has persisted in major metropolitan
areas, especially in the Northeast and Midwest and a large share of
minorities still live in neighborhoods with virtually no White residents
(Logan 2013).

A theory of ``place stratification'' is a better fit, incorporating
discriminating institutions that limit residential movement of African
Americans into White neighborhoods, such as biased residential
preferences among nonHispanic Whites and discrimination in the real
estate market (Charles 2003; Krysan et al. 2009; Turner et al. 2013).

Yet, for many at the lower end of the economic spectrum, stability means
imprisonment: even though many families have left, researchers estimate
that some 70\% of families in today's impoverished neighborhoods were
living there in the 1970s as well (Sharkey 2012).

Scholars writing on the ``geographies of opportunity'' (Briggs 2005)
argue that the spatial relationships between high quality housing, jobs,
and schools structure social mobility. Patterns of urban development in
the United States have resulted in uneven geographies of opportunity, in
which low-income and families of color experience limited access to
affordable housing, high quality schools, and good-paying jobs.

\subsection{A Review of Mass Appraisal
Techniques}\label{a-review-of-mass-appraisal-techniques}

Much of the research on predicting real estate values has been in
service of creating mass appraisal models. Mass appraisal models share
many characteristics with predictive machine learning model modeling.
Mass appraisal models are data-driven, standardized methods that employ
statistical testing (Eckert 1990).

(Quintos 2013) Attempts to measure latent variables through a random
effect regression model to predict income and expense of non-filers.
Difference between the Assessed Value and the Market Value.

New York City annually values commercial properties by the income
approach. Commercial properties with an assessed value greater than
\$40,000 are required to file income and expense statements with the
Department of Finance. Some of these required filers may apply for
exclusion from filing or they may choose not to file and instead pay a
penalty. There are also voluntary filers, who are not required to file
but nevertheless submit statements. The filings received are used to
formulate income and expense regression models. These models are used to
develop comparable rental models and to formulate assessment guidelines
based on location and physical characteristics.

For models of income and expense, however, we are not aware of a model
in a random effects (panel data) framework---most likely due to the lack
of property-level data of income and expense filings.

(d'Amato 2017) Great Lit review in first chapter on the evolution of the
Automated Valuation Model. Walks through all different kinds of spatial
models: OLS, Heirarchichal, spatial lag, spatial error, etc. Explains
COD (coeficient of dispersion). Dodd Frank Act implements financial
regulatory reform after the financial crisis of 2008. In particular the
title XIV subtitle F distinguishes appraisal process from automated
valuation modelling, reorganizing both. In particular it was stressed
how the role of valuation (appraisal) cannot be replaced by AVM. Our
point of view is coherent with the Dodd Frank act (and Appraisal Methods
and the Non-Agency Mortgage Crisis 29 thereby also Pugh's view but not
Woodward's): automated valuation modelling is increasingly adaptable in
describing real estate market behaviour without succeeding in replacing
local information and human inspection in the valuation (appraisal)
procedure.

(Koschinsky 2012) This is a recent and thorough discussion of parametric
hedonic regression techniques. Some of the variables included are
derived from nearby properties, similar to my technique, and these
variables are found to be predictive. Methodology section (2) contains a
brief but robust literature review of hedonic price modeling applied to
real estate marginal willingness to pay (MWTP) for locational
attributes. The basic hedonic model assumes that the utility of a
household or an individual is a function of a composite good x; a vector
of structural characteristics S; a vector of social and neighborhood
characteristics N; and finally a vector of locational characteristics L.
This study adds to a small body of existing literature that extends this
research by addressing the valuation of a property's locational
attributes from a spatial perspective.

In the model, for a spatial lag, they use a ``We specify W as a queen
contiguity weights matrix.'' The second set of locational attribute data
represents a new way of measuring attributes of neighboring properties
that is fully exogenous since it is derived from a different dataset
than the sales data: It is based on structural characteristics of all
residential properties built before 1997 that are not for sale but are
within 1,000 feet of a 1997 sale. \ldots{} The variables included for
neighboring properties within 1,000 feet of a sale are average age, poor
condition (\%), with electric heating source (strongly correlated with
older age) (\%), poor construction grade (1--5) (\%), high construction
grade (10--13) (\%), and detached single-family homes (\%). The spatial
parameter lambda is positive and significant in all cases, i.e.~the
relation between a home's price and the average price of its neighboring
homes is characterized by positive spatial autocorrelation where, for
instance, high-price homes are surrounded by houses with high prices

In short, for the data in this study locational characteristics are
valued at least as much as (if not more) than important structural
characteristics.

In this case the correct welfare measure should be the direct effect
since there is a strong argument in the literature (e.g.~Pace and Gilley
1998) that spatial autocorrelation in house prices is related to the
practice of realtors, appraisers and home owners of using nearby
comparable sales to determine the sales price of a property. Therefore
it is to be expected that a house which is in a neighborhood where the
sales price of recently sold houses is high will be higher than a
similar house surrounded by houses recently sold at a low price. This
will lead to spatial autocorrelation in house prices, but the origin for
such autocorrelation is a pecuniary externality

(Fotheringham 2015) Explores the use of GWR to forecast prices.Explores
the combination of time-series forecasting (in the Holt-Winters
tradition) to geogaphically weighted regression (GWR). GWR is a
variation on OLS that allows for ``adaptive bandwidths'' of local data
to be included, i.e., for each estimate, the number of data points
included varies (optimized using CV). In addition, the data points are
weighted according to distance. This is known as a ``local'' model

\subsection{Gentrification and Neighborhood
Ascent}\label{gentrification-and-neighborhood-ascent}

(Zuk 2015) The primary concern of gentrification is one of its negative
outcomes: displacement

\subsection{Has Machine Learning Been Applied to this Problem
Before?}\label{has-machine-learning-been-applied-to-this-problem-before}

(Zuk 2015) Urban simulation models are guided by consumer
decision-making, rather than the development decisions -- flows of
people rather than capital -- and have neglected the role of race; thus
they may not capture complex gentrification dynamics.

Presentation by researchers from the Urban Institute (Austin Turner and
Snow 2001). Analyzing data for the DC area, they identified the
following five leading indicators as predictive of future gentrification
(defined as sales prices that are above the D.C. average) as low priced
areas that are: 1) adjacent to higher-priced areas, 2) have good metro
access, 3) contain historic architecture, 4) have large housing units,
and 5) experience over 50\% appreciation in sales prices between 1994
and 2000.

Census tracts were scored for each indicator and then ranked according
to the sum of indicators with a maximum value of 5. (Note: Analysis done
at the census-tract level)

(Chapple 2009). Chapple adopted Freeman's (2005) definition of
gentrifying neighborhoods as lowincome census tracts in central city
locations in 1990 that by 2000 experienced housing appreciation and
increased educational attainment above the 9-county regional average.
(Note: Analysis done at the census-tract level)

(Pollack, Bluestone, and Billingham 2010). Analyzing 42 neighborhoods
(block groups within ½ mile of a transit station) near rail stations in
12 metro areas across the United States, they studied changes between
1990 and 2000 for neighborhood socioeconomic and housing characteristics
(Note: Analysis done at the neighborhood level)

(Schernthanner H. 2016) Paper compares traditional linear regression
techniques to more advnaced techniques such as krigging (stochastic
interpolation) and random forrest; finds that more advanced techniques
are sound and more accurate. The research findings indicate that the
analysis results achieved by any of the new methods, ranging from
stochastic interpolation to the ``random forest'' method of machine
learning, are more valid than results obtained from traditional
statistical methods

(Guan et al. 2014) Uses three different approaches to defining comps,
all using euclidean distance; a radius technique, a k-nearest neighbors
technique using only distance and a k-nearest neighbors technique using
all attributes. Interestingly, the location-only KNN neighborhood
performed best, although by a very slim margin (potentially
meaningless). The MRA {[}Multiple Regression Analysis{]} method,
although widely used in mass appraisal, has been criticized for its
inability to model data features typically found in real estate data.
Common problems with MRA assessment of real estate properties are well
known and they include nonlinear- ity, multicollinearity, and
heteroscedasticity (Antipov and Pokryshevskaya 2012; Kilpatrick 2011;
Mark and Goldberg 1988; Peterson and Flanagan 2009). In recent years,
data mining methods have been proposed as an alternative, and have been
tested with very mixed results.

(Fu 2014) Prediction model for real estate in Beijin, China. They do a
clustering, then do a rank-ordered prediction of investment returns
segmented into categories:
4\textgreater{}3\textgreater{}2\textgreater{}1\textgreater{}0

While a number of estate appraisal methods have been devel- oped to
value real property, the performances of these meth- ods have been
limited by the traditional data sources for es- tate appraisal

the geographic dependencies of the value of an estate can be from the
characteristics of its own neighborhood (individual), the values of its
nearby estates (peer), and the prosperity of the affiliated latent
business area (zone)

ClusRanking is able to exploit geographic individ- ual, peer, and zone
dependencies in a probabilistic ranking model. Specifically, we first
extract the geographic utility of estates from geography data, estimate
the neighborhood popularity of estates by mining taxicab trajectory
data, and model the influence of latent business areas via ClusRank-
ing.

From related works: Recent works {[}8, 21{]} study the automated
valuation models, which aggregate and ana- lyze physical characteristics
and sales prices of comparable properties to provide property valuations

(Rafiei 2016) Fascinating paper which employs a Restricted Boltzmann
Machine (neural network with back propagation) to predicted the sale
price of residential condos in Tehran, Iran. The paper focuses on
computational efficiency. A non-mating genetic algorithm is used for
dimensionality reduction. The paper concludes that two primary
strategies help in this regard: Sales which happened closer in time to a
prediction are more important, and it also uses a learner to accelerate
the recognition of important features. The paper compares this technique
to several other common NN approaches and finds that while not
necessarily the only way to get the best answer, it is definitely the
fastest way to get to the best answer. The lit review sections walks
through several recent and notable papers specifically on the topic of
sales price prediction of real estate. There is also mention of a paper
which characterizes a real estate market as supply inelastic which may
be worth investigating further.

(Helbich 2013) This is a very recent paper which contains a brief but
robust literature review in the introduction. Great quote: hedonic
pricing models ``can be improved in two ways: (a) Through novel
estimation techniques (e.g.~Brunauer et al., 2010; Koschinsky,
Lozano-Gracia, \& Piras, 2011) and (b) by ancillary structural,
locational, and neighborhood variables on the basis of Geographic
Information System (GIS) algorithms (e.g.~Hamilton \& Morgan, 2010)''

Let's follow up on the sources mentioned. I believe my
micro-neighborhood technique falls into the ``unique estimation''
bucket, so it would be wise to position it that way

(Kontrimasa 2011) Mass appraisal is commonly used to compute real estate
tax. Study uses an n = 100 (very small) and compares accuracy of linear
regression vs other ANN techniques like SVM.

(Dietzell 2014) This paper examines internet search query data provided
by ``Google Trends'', with respect to its ability to serve as a
sentiment indicator and improve commercial real estate forecasting
models for transactions and price indices. The empirical results show
that all models augmented with Google data, combining both macro and
search data, significantly outperform baseline models which abandon
internet search data

(Gary and D. 2011) Examines the effects of walkability on property
values and investment returns. Use data from the National Council of
Real Estate Investment Fiduciaries and Walk Score to examine the effects
of walkability on the market value and investment returns of more than
4,200 office, apartment, retail and industrial properties from 2001 to
2008 in the United States. On a 100-point scale, a 10-point increase in
walkability increased values by 1--9\%, depending on property type. We
also found that walkability was associated with lower cap rates and
higher incomes, suggesting it has been favored in both the capital asset
and building space markets

(Park 2015) Machine learning applied to residential real estate price
prediction. Developed a housing price prediction model based on machine
learning algorithms such as C4.5, RIPPER, Naïve Bayesian, and AdaBoost
and compare their classification accuracy performance. The experiments
demonstrate that the RIPPER algorithm, based on accuracy, consistently
outperforms the other models in the performance of housing price
prediction.

\subsection{sample citations}\label{sample-citations}

Sample Citation: (Antipov and Pokryshevskaya 2012) (see: Antipov and
Pokryshevskaya 2012, 33--35; also Antipov and Pokryshevskaya 2012, ch.~1
and \emph{passim})

A minus sign (-) before the @ will suppress mention of the author in the
citation. This can be useful when the author is already mentioned in the
text:

Antipov says blah (2012).

You can also write an in-text citation, as follows:

Antipov and Pokryshevskaya (2012) says blah.

\section{Methodology}\label{methodology}

\subsection{Data}\label{data}

\subsection{Algorithm}\label{algorithm}

Random Forrest has several advantages over traditional geographic
weighted regression, amoung them:

\begin{enumerate}
\def\labelenumi{\arabic{enumi}.}
\tightlist
\item
  Ability to handle large amounts of categorical data without much
  pre-processing
\item
  Ability to model in spite of missing values in data
\item
  Eliminated colinearity as a concern
\item
  Allows for the introduction of many more variables without requiring
  penalty for additional predictors
\item
  Works relatively fast and can be parallelized
\end{enumerate}

\subsection{Model Diagnostics}\label{model-diagnostics}

\section{Results}\label{results}

\subsection{Probability of Sale Model}\label{probability-of-sale-model}

\subsection{Sale Price Model}\label{sale-price-model}

\subsection{Using the Models in
Practice}\label{using-the-models-in-practice}

\section{Conclusions and Future
Research}\label{conclusions-and-future-research}

\subsection{Future Research}\label{future-research}

\subsection{Conclusion}\label{conclusion}

\section*{References}\label{references}
\addcontentsline{toc}{section}{References}

\hypertarget{refs}{}
\hypertarget{ref-Almanie2015}{}
Almanie, R.; Lor, T.; Mirza. 2015. ``Crime Prediction Based on Crime
Types and Using Spatial and Temporal Criminal Hotspots.''
\emph{International Journal of Data Mining \& Knowledge Management
Process (IJDKP)} 5 (4).

\hypertarget{ref-antipov12}{}
Antipov, Evgeny A., and Elena B. Pokryshevskaya. 2012. ``Mass Appraisal
of Residential Apartments: An Application of Random Forest for Valuation
and a Cart-Based Approach for Model Diagnostics.'' \emph{Expert Systems
with Applications}.

\hypertarget{ref-Chapple2016}{}
Chapple, Miriam, Karen; Zuk. 2016. ``Forewarned: The Use of Neighborhood
Early Warning Systems for Gentrification and Displacement.''
\emph{Cityscape: A Journal of Policy Development and Research} 18 (3).

\hypertarget{ref-Dietzell2014}{}
Dietzell, Nicole; Schäfers, Marian Alexander; Braun. 2014.
``Sentiment-Based Commercial Real Estate Forecasting with Google Search
Volume Data.'' \emph{Journal of Property Investment \& Finance,} 32 (6):
540--69.

\hypertarget{ref-Springer2017}{}
d'Amato, Tom, Maurizio; Kauko, ed. 2017. \emph{Advances in Automated
Valuation Modeling}. Springer International Publishing.

\hypertarget{ref-Eckert1990}{}
Eckert, J. K. 1990. \emph{Property Appraisal and Assessment
Administration}. Chicago, IL.: International Association of Assessing
Officers.

\hypertarget{ref-Fotheringham2015}{}
Fotheringham, R; Yao, A.S.; Crespo. 2015. ``Exploring, Modelling and
Predicting Spatiotemporal Variations in House Prices.'' \emph{The Annals
of Regional Science} 54.

\hypertarget{ref-Fu2014}{}
Fu, Yanjie; et al. 2014. \emph{Exploiting Geographic Dependencies for
Real Estate Appraisal: A Mutual Perspective of Ranking and Clustering}.
Proceedings of the 20th ACM SIGKDD international conference on Knowledge
discovery; data mining.

\hypertarget{ref-Pivo2011}{}
Gary, Pivo, and Fisher Jeffrey D. 2011. ``The Walkability Premium in
Commercial Real Estate Investments.'' \emph{Real Estate Economics} 39
(2): 185--219.
doi:\href{https://doi.org/10.1111/j.1540-6229.2010.00296.x}{10.1111/j.1540-6229.2010.00296.x}.

\hypertarget{ref-Geltner2017}{}
Geltner, David, and Alex Van de Minne. 2017. ``Do Different Price Points
Exhibit Different Investment Risk and Return Commercial Real Estate.''
Real Estate Research Institute.

\hypertarget{ref-Guan2014}{}
Guan, Jian, Donghui Shi, Jozef M. Zurada, and Alan S. Levitan. 2014.
``Analyzing Massive Data Sets: An Adaptive Fuzzy Neural Approach for
Prediction, with a Real Estate Illustration.'' \emph{Journal of
Organizational Computing and Electronic Commerce} 24 (1). Taylor \&
Francis: 94--112.
doi:\href{https://doi.org/10.1080/10919392.2014.866505}{10.1080/10919392.2014.866505}.

\hypertarget{ref-Helbich2013}{}
Helbich, et al., Marco. 2013. ``Boosting the Predictive Accuracy of
Urban Hedonic House Price Models Through Airborne Laser Scanning.''
\emph{Computers, Environment and Urban Systems} 39: 81--92.

\hypertarget{ref-Johnson2007}{}
Johnson, Ken, Justin Benefield, and Jonathan Wiley. 2007. ``The
Probability of Sale for Residential Real Estate.'' \emph{Journal of
Housing Research} 16 (2): 131--42.
doi:\href{https://doi.org/10.5555/jhor.16.2.0234g75800h5k8x6}{10.5555/jhor.16.2.0234g75800h5k8x6}.

\hypertarget{ref-Kontrimasa2011}{}
Kontrimasa, Antanas, Vilius; Verikasb. 2011. ``The Mass Appraisal of the
Real Estate by Computational Intelligence.'' \emph{Applied Soft
Computing}.

\hypertarget{ref-Koschinsky2012}{}
Koschinsky, J. et al. 2012. ``The Welfare Benefit of a Home's Location:
An Empirical Comparison of Spatial and Non-Spatial Model Estimates.''
\emph{Journal of Geographical Systems} 10109.

\hypertarget{ref-Miller2015}{}
Miller, J.; Aspinall, J.; Franklin. 2007. ``Incorporating Spatial
Dependence in Predictive Vegetation Models.'' \emph{Ecological
Modelling} 202 (3): 225--42.

\hypertarget{ref-Park2015}{}
Park, Jae Kwon, Byeonghwa; Bae. 2015. ``Using Machine Learning
Algorithms for Housing Price Prediction: The Case of Fairfax County,
Virginia Housing Data.'' \emph{Expert Systems with Applications} 42 (6):
2928--34.

\hypertarget{ref-Quintos2013}{}
Quintos, Carmela. 2013. ``Estimating Latent Effects in Commercial
Property Models.'' \emph{Journal of Property Tax Assessment \&
Administration} 12 (2).

\hypertarget{ref-Rafiei2016}{}
Rafiei, Hojjat, Mohammad Hossein; Adeli. 2016. ``A Novel Machine
Learning Model for Estimation of Sale Prices of Real Estate Units.''
\emph{Journal of Construction Engineering and Management} 142 (2).

\hypertarget{ref-Schernthanner2016}{}
Schernthanner H., Gonschorek J., Asche H. 2016. ``Spatial Modeling and
Geovisualization of Rental Prices for Real Estate Portals.''
\emph{Computational Science and Its Applications} 9788.

\hypertarget{ref-urban2016}{}
Solomon Greene, Molly Scott, Rolf Pendall, and Serena Lei. 2016. ``Open
Cities: From Economic Exclusion to Urban Inclusion.'' \emph{Urban
Institue Brief}, June. Urban Institue Brief.

\hypertarget{ref-Zuk2015}{}
Zuk, Miriam; et al. 2015. ``Gentrification, Displacement and the Role of
Public Investment: A Literature Review.''


\end{document}
